\documentclass[]{cv}
\usepackage{lmodern}
\usepackage{amssymb,amsmath}
\usepackage{ifxetex,ifluatex}
\usepackage{fixltx2e} % provides \textsubscript
\ifnum 0\ifxetex 1\fi\ifluatex 1\fi=0 % if pdftex
  \usepackage[T1]{fontenc}
  \usepackage[utf8]{inputenc}
\else % if luatex or xelatex
  \ifxetex
    \usepackage{mathspec}
  \else
    \usepackage{fontspec}
  \fi
  \defaultfontfeatures{Ligatures=TeX,Scale=MatchLowercase}
\fi
% use upquote if available, for straight quotes in verbatim environments
\IfFileExists{upquote.sty}{\usepackage{upquote}}{}
% use microtype if available
\IfFileExists{microtype.sty}{%
\usepackage{microtype}
\UseMicrotypeSet[protrusion]{basicmath} % disable protrusion for tt fonts
}{}
\usepackage{hyperref}
\PassOptionsToPackage{usenames,dvipsnames}{color} % color is loaded by hyperref
\hypersetup{unicode=true,
            colorlinks=true,
            linkcolor=black,
            citecolor=Blue,
            urlcolor=black,
            breaklinks=true}
\urlstyle{same}  % don't use monospace font for urls


\IfFileExists{parskip.sty}{%
\usepackage{parskip}
}{% else
\setlength{\parindent}{0pt}
\setlength{\parskip}{6pt plus 2pt minus 1pt}
}
\setlength{\emergencystretch}{3em}  % prevent overfull lines
\providecommand{\tightlist}{%
  \setlength{\itemsep}{0pt}\setlength{\parskip}{0pt}}
\setcounter{secnumdepth}{0}
% Redefines (sub)paragraphs to behave more like sections
\ifx\paragraph\undefined\else
\let\oldparagraph\paragraph
\renewcommand{\paragraph}[1]{\oldparagraph{#1}\mbox{}}
\fi
\ifx\subparagraph\undefined\else
\let\oldsubparagraph\subparagraph
\renewcommand{\subparagraph}[1]{\oldsubparagraph{#1}\mbox{}}
\fi

\date{}

\begin{document}

\subsection{Expériences}\label{expuxe9riences}

\begin{description}
\tightlist
\item[Depuis 01/2015]
\textbf{Ingénieur de recherche, expert en technologies GPU}, Centre de
calcul de Champagne-Ardenne ROMEO, Université de Reims
Champagne-Ardenne, Reims.
\end{description}

\begin{itemize}
\item
  \begin{itemize}
  \tightlist
  \item
    \textbf{Laboratoire d'applications GPU}: Expertise auprès des
    utilisateurs pour le portage et l'optimisation de codes de calcul
    GPU.
  \item
    \textbf{Veille techologique}: Mise en œuvre de librairies, outils et
    logiciels spécifiques à l'exploitation du calculateur.
  \item
    \textbf{Communication}: Formations aux utilisateurs et étudiants en
    technologies de calcul sur GPU et en calcul hybride. Participation à
    la production de communications scientifiques dans le cadre des
    projets.
  \end{itemize}
\end{itemize}

\begin{description}
\tightlist
\item[10/2011 -- 12/2014]
\textbf{Doctorat}: \emph{Couplage de modèles, algorithmes multi-échelles
et calcul hybride}, Laboratoire Jean Kuntzmann, Université Joseph
Fourier, Grenoble, Direction: G.-H. Cottet et C. Picard
\end{description}

\begin{itemize}
\item
  \begin{itemize}
  \tightlist
  \item
    \textbf{Recherche}: Analyse et extension d'une méthode particulaire
    avec remaillage, implémentation d'algorithmes multi-échelles sur
    GPU, calcul hybride multi-CPU/multi-GPU sur clusters du mésocentre
    grenoblois CIMENT et serveurs du laboratoire, applications à la
    résolution d'écoulements turbulents.
  \item
    \textbf{Label C3I:} Certificat de Compétences en Calcul Intensif,
    GENCI.
  \item
    \textbf{Enseignement}: Méthodes informatiques et techniques de
    programmation (135 h, Cours/TD/TP, L1), fonctions de plusieurs
    variables et calcul matriciel (36 h, TD, L2), expérimentations sur
    Scilab: image, signal, interpolation, courbes et surfaces (27 h, TP,
    L1).
  \item
    \textbf{Communication}: Conférences nationale
    \autocite{Etancelin2013b} et internationales
    \autocite{Etancelin2014a} \autocite{Balarac2014}, article de revue
    avec comité de relecture \autocite{Etancelin2014}, documentation
    technique utilisateur et développeur.
  \end{itemize}
\end{itemize}




\setlength\bibitemsep{1.3\itemsep}
%\nocite{*}
\renewcommand{\mkbibnamefirst}[1]{\normalfont{#1}}
\renewcommand{\mkbibnamelast}[1]{\normalfont{#1}}
\renewcommand{\mkbibnameprefix}[1]{\normalfont{#1}}
\renewcommand{\mkbibnameaffix}[1]{\normalfont{#1}}
\printbibliography{}

\end{document}
